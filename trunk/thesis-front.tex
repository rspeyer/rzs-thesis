\begin{thesistitlepage}               %% Generate the title page.
\end{thesistitlepage}

%\begin{thesiscopyrightpage}                 %% Generate the copyright page.
%\end{thesiscopyrightpage}

\begin{thesisabstract}
Webcams are cheap sensors that capture a potentially large amount of information about a scene. This thesis considers the use of regression and correlation techniques such as Canonical Correlation Analysis (CCA) to convert these webcams into environmental sensors and predict the values of weather signals. Local environmental properties often directly affect the images we collect from the webcams; whether it is cloudy or sunny is visible by the presence of shadows; wind speed and direction is visible in smoke, flags, or close up views of trees; particulate density is reflected in haziness and the color spectrum during sunset. Using the AMOS database, which has been archiving nearly 1,000 webcams every 30 minutes for the last 3 years, we explore relationships between the amount of training data and the accuracy with which we are able to infer the values of certain weather signals including wind speed \& direction and vapor pressure from inherent properties in the image. This allows the webcams \textit{already} installed across the earth to act as generic sensors to improve our understanding of local weather patterns and variations.
\end{thesisabstract}

\begin{thesisacknowledgments}
First and foremost, I would like to thank Dr. Robert Pless, who has been my research and faculty adviser for the last four years. His constant attention and guidance have been crucial in the construction of this thesis and I am extremely grateful for all of his help. My experience working under him in the Media \& Machines Lab is one of the most valuable of my college career and has taught me skills that I will surely use after I graduate.

I would also like to thank Nathan Jacobs and many other students who helped review this thesis and the related research as well as Dr. William Smart and Dr. Ron Cytron for sitting on my Thesis Committee along with Dr. Pless.

Finally, I would like to thank the Department of Computer Science \& Engineering for promoting and providing many opportunities for undergraduates to get involved in research projects. I have benefited greatly from working in a research lab during my time here and that would not have been possible without the department's commitment to undergraduate research.
\end{thesisacknowledgments}

\begin{singlespace}
\tableofcontents
%\listoftables
\listoffigures
\end{singlespace}